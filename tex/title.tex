\tunemarkup{coverimage}{%
\newcommand{\adjustimg}{\checkoddpage\ifoddpage\hspace*{\dimexpr\evensidemargin-\oddsidemargin}\else\hspace*{-\dimexpr\evensidemargin-\oddsidemargin}\fi}
\newcommand{\centerimg}[2][width=\textwidth]{\makebox[\textwidth]{\adjustimg\includegraphics[#1]{#2}}}
\newpagecolor{ubpagecolor}\afterpage{\restorepagecolor}
\includepdf{images/British-Study-Edition-Cover-tiny.png}
}

\makeatletter
\bib@raise@anchor{\bibpdfbookmark[0]{Титульный лист}{Ttl}}%
\makeatother

\vspace*{\stretch{0.1}}
\begin{center}
{
\bibcovertitlefont
\titlefontsize
ՀԻՆԳԵՐՈՐԴ ԴԱՐԱՇՐՋԱՆԱՅԻՆ\\
ՀԱՅՏՆՈՒԹՅՈՒՆ\\
\intertitlefontsize
\bibemph{որը այլ կերպ կոչվում է}\\
\titlefontsize
ՈՒՐԱՆՏԻԱՅԻ ԹՂԹԵՐ\\
}%
{%
\vspace*{\stretch{0.1}}
\bibemph{Թարգմանված է անգլերենից}\\
\bibemph{Գլխավոր խմբագիր Տիգրան Այվազյան}\\
}%
\vspace*{\stretch{0.4}}
\includegraphics[width=0.2\columnwidth]{images/Phoenix-Logo-Circles.jpg}\\
\vspace*{\stretch{0.3}}
\titlesepbig\\
\vspace*{\stretch{0.1}}
\end{center}

\titleframe

\newpage

%\newcommand{\serpimolot}{{\fontspec{Mortbats} K}}

\begin{center}
\vspace*{\stretch{0.3}}
\begin{center}\shadowbox{\strut\parbox{0.7\linewidth}{\normalsize\bfseries\itshape Մարդկային կյանքի ամենակարևոր բաներից մեկը պարզելն է, թե ինչին է հավատացել Հիսուսը, հայտնաբերել նրա իդեալները և ձգտել իր վեհ կյանքի նպատակի իրականացմանը: \bibref[\latintext(196:1.3)]{p196 1:3}}}\end{center}
\vspace*{\stretch{0.6}}
\tunemarkup{pgnexus10}{\fontsize{11}{15}\itshape}
\parbox{0.9\linewidth}{\centering
Ուղարկել բոլոր մեկնաբանությունները այստեղ {\makeatletter\latintext\upshape\bfseries aivazian.tigran@gmail.com\makeatother}\\[1ex]
\tux\ Գրանշան {\latintext\bfseries\XeLaTeX\ (\TeX\ Live 2017) Linux} համակարգում\\[4pt]
\upshape\normalsize\bfseries Տարբերակ: \tunemarkup{pgnexus10}{{\latintext 10"} գունավոր}\tunemarkup{pgthinmob}{20:9 գունավոր}\\
\upshape\bfseries Ամսաթիվ: \mytoday{}\\
}
\vspace*{\stretch{0.1}}
\end{center}

\titleframe
